\documentclass[a4paper,11pt]{article}
\usepackage{a4wide,amsmath,ngerman,url,graphicx}
\usepackage[utf8]{inputenc}
\parskip4pt
\parindent0pt

\newcommand{\br}[1]{\left(#1\right)}
\newcommand{\erf}{\mathrm{erf}}
\newcommand{\m}{\cdot}

\title{Analyse der Coronastatistiken. Teil 2} 
\author{Hans-Gert Gräbe, Leipzig}
\date{Version vom 13. April 2020}

\begin{document}
\maketitle

Dieser Text ist eine Fortschreibung des ersten Teils. Die dortigen
Beschreibungen der allgemeinen Rahmenbedingungen werden als bekannt
vorausgesetzt. 

\section{Logistische Funktion}

Generell ist ein Modell auf der Basis einer \emph{Logistischen Funktion}
\begin{gather*}
  u(t)=\frac{K}{1+C\m\exp(-rt)}\tag{L.1}
\end{gather*}
die anerkanntere Form der Modellierung der Ausbreitung einer Infektion, siehe
dazu den entsprechenden Wikipedia-Eintrag. 
\begin{center}
  \includegraphics[width=.8\textwidth]{LC.png}\\[1em]
  \begin{quote}
    {Logistische Kurve $u(t)=\frac{1}{1+12\exp(-2t)}$ (blau) sowie deren erste
      (rot) und zweite Ableitung (grün) Ableitung}
  \end{quote}
\end{center}
$K$ steht dabei für die Sättigungsgrenze $\lim_{t\to\infty}{u(t)}$ und $C$ ist
üblicherweise als $C=\frac{K}{u(0)}-1$ angeschrieben, was sich unmittelbar aus
der Umstellung der Formel für $u(0)$ nach $C$ ergibt. Der Wendepunkt dieser
Funktion und damit das Maximum der ersten Ableitung liegt als Nullstelle der
zweiten Ableitung bei $t_0=\frac{\log(C)}{r}$, was sich unmittelbar mit Maxima
berechnen lässt.  Dies ist gerade der Median der Kurve, sollte also mit dem
oben berechneten $m$ zusammenfallen.  

Derartige Funktionen lassen sich aber deutlich schlechter schätzen als
Funktionen, die sich wie oben auf einfache Weise auf einen polynomialen
Zusammenhang reduzieren lassen, da sie inhärent transzendent sind.  Siehe
hierzu aber die Arbeit von (Engel 2010) und die Modellierung mit
\textsc{GeoGebra} in (Elschenbroich 2020).

In (Engel 2010) wird insbesondere darauf hingewiesen, dass sich mit einer
guten Schätzung von $K$ die anderen beiden Parameter mit einem linearen
Fitting bestimmen lassen. Wir setzen dazu $C=\exp(c)$, womit sich Formel
$(L.1)$ zu
\begin{gather*}
  \log\br{\frac{K}{u(t)}-1}=c-rt  \tag{L.2}
\end{gather*}
umstellen lässt.  Der Parameter $K$ ist dabei manuell zu schätzen, so dass die
gefittete Kurve möglichst gut auf die Daten passt.

Im Skript ist das Ganze in einer Funktion \texttt{lFit(G,K0)} implementiert,
der eine Liste $G$ zu übergeben ist, in der vorab alle Datenpunkte mit $y_t\le
10$ ausgefiltert werden.  Es ergeben sich folgende Schätzungen für die Zahl
der (in der Statistik erfassten) infizierten Personen (Stand 10.04.2020):
\begin{center}
  \begin{tabular}{|l|c|c|c|c|}\hline
    Land & $K$ & $c$ & $r$ & $c/r$ \\\hline
    Deutschland & 125000 & 18.058 & 0.198 & 91.11\\
    Italien & 145000 & 17.445 & 0.207 & 84.47\\
    Österreich & 13000 & 23.053 & 0.268 & 85.90 \\
    Spanien & 160000 & 23.815 & 0269 & 88.42 \\
    China (Hebei) & 70000 & 4.664 & 0.103 & 45.12\\\hline
  \end{tabular}
\end{center}

\section{Die „Verdoppelungsdebatte“}

Anfang April 2020 kommt eine Diskussion hoch, dass man die rigiden
Beschränkungen erst aufheben könne, wenn „die Verdopplungszeit der Infektionen
größer als 14 Tage“ sei.

So meldet zum Beispiel der Deutschlandfunk am 04.04.2020\footnote{\raggedright
  \url{https://www.deutschlandfunk.de/covid19-verdopplungszeit-der-coronavirus-infektionen-in.1939.de.html?drn:news_id=1117169}}
\begin{quote}
  Die Verdopplungszeit der Ausbreitung von Coronavirus-Infektionen in
  Deutsch"|land hat sich in den vergangenen Tagen verlangsamt.

  Für ganz Deutschland liegt sie nun bei 11{,}2 Tagen. Die Lage in den
  Bundesländern ist unterschiedlich. In den großen Flächenländern liegt die
  Verdopplungszeit in Nordrhein-Westfalen bei 13{,}1 Tagen, in
  Baden-Württemberg bei 12{,}5 Tagen und in Bayern bei 9{,}7 Tagen. In Berlin
  sind es inzwischen 12{,}8 Tage, in Hamburg 12{,}4. Im Saarland hingegen
  liegt die Verdoppelungszeit bei 5{,}5 Tagen, in Sachsen bei 11{,}0 Tagen.
\end{quote}
Auch wenn dies nicht immer deutlich wird, bezieht sich die Verdopplungszeit
$v_t$ auf die kumulierten Daten und steigt deshalb bereits durch die schiere
Masse der Infizierten.  Ist $y(t)=mt+n$ ein linearer Zusammenhang, so ergibt
sich $v_t=\frac{y(t)}{m}$. Für einen annähernd linearen Zusammenhang kann man
also $v_t=\frac{y(t)}{y'(t)}$ als Schätzung nehmen.  Die Zahl lässt sich auch
aus unseren Daten leicht berechnen: Ist $y_t$ die kumulierte Zahl der
Infizierten am Tag $t$ und $d_t$ die Zahl der Neuinfektionen, so ist nach
$v_t=\frac{y_t}{d_t}$ Tagen eine Verdopplung der Infizierten erreicht, die
Zuwachsrate $d_t$ über diesen Zeitraum als konstant vorausgesetzt.  Beide
Datenreihen (Stand 10.04.2020) hatten wir schon oben extrahiert, so dass wir
eine einfache Funktion \texttt{doublePlot(Land)} schreiben können, um die
folgenden Plots zu erzeugen:
\begin{center}  
  \begin{minipage}{.33\textwidth}\centering
    \includegraphics[width=\textwidth]{Italy-DP.png}\\[1em] {Italien}
  \end{minipage}\hfill
  \begin{minipage}{.33\textwidth}\centering
    \includegraphics[width=\textwidth]{Germany-DP.png}\\[1em] {Deutschland}
  \end{minipage}\hfill
  \begin{minipage}{.33\textwidth}\centering
    \includegraphics[width=\textwidth]{Austria-DP.png}\\[1em] {Österreich}
  \end{minipage}
\end{center}


\section{Literatur}

\begin{itemize}
\item Hans-Jürgen Elschenbroich. Corona: Mathematik \& Modellbildung.\\
  \url{https://www.geogebra.org/m/cfammtpe}.  2020.
\item Joachim Engel. Parameterschätzen in logistischen Wachstumsmodellen.
  Stochastik in der Schule 30 (2010) 1, S. 13–18.
\end{itemize}

\end{document}
