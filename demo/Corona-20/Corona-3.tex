\documentclass[a4paper,11pt]{article}
\usepackage{a4wide,amsmath,ngerman,url,graphicx}
\usepackage[utf8]{inputenc}
\parskip4pt
\parindent0pt

\newcommand{\br}[1]{\left(#1\right)}
\newcommand{\erf}{\mathrm{erf}}
\newcommand{\m}{\cdot}

\title{Analyse der Coronastatistiken. Teil 3} 
\author{Hans-Gert Gräbe, Leipzig}
\date{Version vom 03.06.2020}

\begin{document}
\maketitle

Dieser Text ist eine Fortschreibung der ersten beiden Teile. Die dortigen
Beschreibungen der allgemeinen Rahmenbedingungen werden als bekannt
vorausgesetzt. 

\section{Die „Verdoppelungsdebatte“}

Anfang April 2020 kommt eine Diskussion hoch, dass man die rigiden
Beschränkungen erst aufheben könne, wenn „die Verdopplungszeit der Infektionen
größer als 14 Tage“ sei.  Maß kann auch hier nur die Zahl der positiv
Getesteten sein. 

So meldet zum Beispiel der Deutschlandfunk am 04.04.2020\footnote{\raggedright
  \url{https://www.deutschlandfunk.de/covid19-verdopplungszeit-der-coronavirus-infektionen-in.1939.de.html?drn:news_id=1117169}}
\begin{quote}
  Die Verdopplungszeit der Ausbreitung von Coronavirus-Infektionen in
  Deutsch"|land hat sich in den vergangenen Tagen verlangsamt.

  Für ganz Deutschland liegt sie nun bei 11{,}2 Tagen. Die Lage in den
  Bundesländern ist unterschiedlich. In den großen Flächenländern liegt die
  Verdopplungszeit in Nordrhein-Westfalen bei 13{,}1 Tagen, in
  Baden-Württemberg bei 12{,}5 Tagen und in Bayern bei 9{,}7 Tagen. In Berlin
  sind es inzwischen 12{,}8 Tage, in Hamburg 12{,}4. Im Saarland hingegen
  liegt die Verdoppelungszeit bei 5{,}5 Tagen, in Sachsen bei 11{,}0 Tagen.
\end{quote}
Auch wenn dies nicht immer deutlich wird, bezieht sich die Verdopplungszeit
$v_t$ auf die kumulierten Daten und steigt deshalb bereits durch die schiere
Masse der positiv Getesteten.  Ist $y(t)=mt+n$ ein linearer Zusammenhang, so
ergibt sich $v_t=\frac{y(t)}{m}$. Für einen annähernd linearen Zusammenhang
kann man also $v_t=\frac{y(t)}{y'(t)}$ als Schätzung nehmen.  Die Zahl lässt
sich auch aus unseren Daten leicht berechnen: Ist $y_t$ die kumulierte Zahl
der positiv Getesteten am Tag $t$ und $d_t$ die Zahl der (getesteten)
Neuinfektionen, so ist nach $v_t=\frac{y_t}{d_t}$ Tagen eine Verdopplung der
positv Getesteten erreicht, die Zuwachsrate $d_t$ über diesen Zeitraum als
konstant vorausgesetzt.  Beide Datenreihen (Stand 10.04.2020) hatten wir schon
oben extrahiert, so dass wir eine einfache Funktion \texttt{doublePlot(Land)}
schreiben können, um die folgenden Plots zu erzeugen:
\begin{center}  
  \begin{minipage}{.33\textwidth}\centering
    \includegraphics[width=\textwidth]{Italy-DP.png}\\[1em] {Italien}
  \end{minipage}\hfill
  \begin{minipage}{.33\textwidth}\centering
    \includegraphics[width=\textwidth]{Germany-DP.png}\\[1em] {Deutschland}
  \end{minipage}\hfill
  \begin{minipage}{.33\textwidth}\centering
    \includegraphics[width=\textwidth]{Austria-DP.png}\\[1em] {Österreich}
  \end{minipage}
\end{center}


\section{Literatur}

\begin{itemize}
\item Hans-Jürgen Elschenbroich. Corona: Mathematik \& Modellbildung.\\
  \url{https://www.geogebra.org/m/cfammtpe}.  2020.
\item Joachim Engel. Parameterschätzen in logistischen Wachstumsmodellen.
  Stochastik in der Schule 30 (2010) 1, S. 13–18.
\end{itemize}

\end{document}
