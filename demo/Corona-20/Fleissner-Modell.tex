\documentclass[a4paper,11pt]{article}
\usepackage{a4wide,amsmath,ngerman,url,graphicx}
\usepackage[utf8]{inputenc}
\parskip4pt
\parindent0pt

\newcommand{\br}[1]{\left(#1\right)}
\newcommand{\erf}{\mathrm{erf}}
\newcommand{\m}{\cdot}

\title{Rekonstruktion des Corona-Modells von Peter Fleißner} 
\author{Hans-Gert Gräbe, Leipzig}
\date{Version vom 22. Mai 2020}

\begin{document}
\maketitle

\begin{abstract}
  Peter Fleißner (Wien) hat umfangreiche eigene
  Modellrechnungen\footnote{Siehe dazu
    \url{http://peter.fleissner.org/Covid/Covid_Access.htm}} zur Dynamik der
  Corona-Pandemie für Österreich veröffentlicht.  Anliegen dieses Texts ist
  es, diese Modellierung nachzuvollziehen und mit den Daten abzugleichen, die
  von der Johns Hopkins University (JHU) regelmäßig veröffentlicht und auf
  github zur Verfügung gestellt werden.
\end{abstract}

\section{Fleißners Modell}

Die Darstellung des Modells ist aus den beiliegenden Folien mit leichten
sprachlichen Präzisierungen meinerseits übernommen. 

Die Gesamtbevölkerung $B$ wird in vier Teile zerlegt: In aktuell Infizierte
$i$ (ungleich der täglich erhobenen Maßzahl der positiv Getesteten $p$),
Genesene $r$, die seit Beginn der Pandemie Verstorbenen $d$ und die
ansteckbaren Personen $s$, die über die Bestandsgleichung $B=i+r+d+s$
verbunden sind.

Die aktuell Infizierten $i$ gehorchen folgender Logik: Ihre Zahl wird durch
die Zahl der neu positiv Getesteten vermehrt ($\Delta(p)$) und durch die Zahl
der Genesenen ($\Delta(r)$) und Verstorbenen ($\Delta(d)$) verringert.  Man
könnte zur Veranschaulichung ein in der Systemdynamik übliches
Badewannenmodell verwenden. Diese Badewanne ist allerdings keine gewöhnliche.
Sie enthält nicht Wasser, sondern die aktuell Infizierten $i(t)$. Die
Änderungsrate $\Delta(i)$ berechnet sich aus der täglichen Zuwachsrate
$\alpha(t)$ der positiv Getesteten $p(t)$ (in summa also $\alpha(t)p(t)$) mal
dem Anteil der noch nicht Angesteckten an den Ansteckbaren
($1-\frac{p}{s}$). Zu Beginn ist dieser Wert gleich 1 und sinkt erst nach
einem längeren Prozess der Durchseuchung.  Aus der Badewanne fließen pro Tag
die Gestorbenen $\Delta(d)$ und die Genesenen $\Delta(r)$.  Damit ergeben sich
die folgenden Beziehungen ($\dot{y}=\frac{d\,y(t)}{d\,t}$):
\begin{align}
  &i=p-r-d,\ s=B-p,\tag{1}\\
  &\dot{p}=\alpha p\left(1-\frac{p}{s}\right),\tag{2}\\
  &\dot{r}=\frac{i}{T_r},\tag{3}\\
  &\dot{d}=\frac{i}{T_d}\tag{4}
\intertext{sowie die Anfangsbedingungen}
  B(0)&=8.9\cdot 10^6,\ p(0)=i(0)=100,\ r(0)=d(0)=0.\tag{5}
\end{align}
HGG: Die beiden Bedingungen in der ersten Zeile sind zusätzlich aufgenommen,
die bei Fleißner aufgeführte weitere Bedingung
$\Delta(i)=\Delta(p)-\Delta(r)-\Delta(s)$ ist damit entbehrlich. 

Die Bevölkerung setzt sich aus der Zahl der Infizierten $i$, der Zahl der
bisher Verstorbenen $d$, der Zahl der von der Infektion genesenen $r$ und der
Zahl der Ansteckbaren $s$ zusammen. Die Bevölkerung wird für die Berechnung
als konstant angenommen (diese Beschränkung kann leicht aufgehoben werden).
Zusätzlich wird die einzige empirisch zugängliche Variable $p$ berechnet, die
Zahl der positiv Getesteten.

$T$ bedeuten die durchschnittlichen Verweildauern der Infizierten, bis sie
sich entweder erholt haben ($T_r$) bzw. verstorben sind ($T_d$), ähnlich wie
die Halbwertszeiten beim radioaktiven Zerfall. Nach der üblichen
englischsprachigen Nomenklatur wäre mein Modell ein SPIRD-Modell (s für
sustainable, p für positively tested, i für infected, r für recovered und d
für death).

Die einzige Quelle empirischer Daten für das Modell waren anfangs die
täglichen Meldungen der Zahl der positiv Getesteten. Sie gehen in die
Ansteckungsraten $\alpha(t)$ ein. Darüber hinaus bestimmt die zentrale
Diffusionsgleichung das dynamische Geschehen und anfänglich grobe Annahmen
über die Verweildauern und Übergangsraten (z.B. von den Infizierten zu den
Genesenen). Die Anfangswerte (100 Infizierte) bestimmen auch im Modell die
Ausgangssituation. Erst nach und nach gab es am Dashboard des Ministeriums
auch Daten für die Toten, Genesenen, Belag der Spitals- und Intensivbetten und
Infizierten.

\section{Kritik der Fleißnerschen Modellierung}

Der verwendete Modellierungsansatz einer \emph{Logistischen Kurve} wird in der
deutschen Wikipedia wie folgt beschrieben: „Die logistische Funktion
beschreibt den Zusammenhang zwischen der verstreichenden Zeit und einem
Wachstum, beispielsweise ... (annähernd) der Verbreitung einer Krankheit im
Rahmen einer Epidemie. Hierzu wird das Modell des exponentiellen Wachstums
modifiziert durch eine sich mit dem Wachstum verbrauchende Ressource“.
Er führt auf eine \emph{Bernoullische Differenzialgleichung}
\begin{gather*}
  f'(t)=k\,f(t)\,(G-f(t))
\end{gather*}
eines „verlangsamten“ exponentiellen Wachstums\footnote{$f'(t)=C\,f(t)$
  beschreibt das exponentielle Wachstum $f(t)=\exp(C\,t)$, der
  Proportionalitätsfaktor $C=k\,(G-f(t))$ nimmt mit der Zeit ab, womit sich
  die Entwicklung verlangsamt. $-k\,f(t)^2$ wirkt dabei als Dämpfungsterm, aus
  naheliegenden Gründen wird die Lösung $f(t)$ nach oben durch die
  Grundgesamtheit $G$ beschränkt.}, das auf eine S-Kurve als Lösung
hinausläuft.

Fleissners Gleichung (2) müsste also in diesem Modell durch
\begin{gather*}
  \dot{p}=k\,p\,(B-p)=(kB)\,p\,\left(1-\frac{p}{B}\right)\tag{2a}
\end{gather*}
ersetzt werden. Unverständlich sind die Beziehungen (3) und (4), da eine
Zustandsübergangsmodellierung $s\to i\to \{d,r\}$ in der Grundgesamtheit $B$
vorgenommen wird. Wenn hier, wie beim radioaktiven Zerfall, mit
Halbwertzeiten, also statistischen Übergangswahrscheinlichkeiten einzelner
„Atome“ (hier also einzelne Menschen), so müsste hier eher auf $i(t-T_r)$ bzw.
$i(t-T_d)$ Bezug genommen werden mit der Zusatzbedingung
$i(t)=r(t+T_r)+d(t+T_d)$, da die Infizierten den Zustand nach der
statistischen Verweildauer definitiv in eine der beiden Richtungen verlassen.
Eine solche Invarianzbedingung fehlt im Modell, kann aber mit Blick auf die
empirische Datenlage, in der drei Zeitreihen $(p,r,d)$ vorliegen, weggelassen
werden.  

\section{Modifizierte Modellierung}

Zu ergänzen.

\end{document}
